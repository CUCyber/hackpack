\section{Auditd}

Linux has a built in auditing framework that acts in kernel space.
This portion of the kernel, communicates with the userspace auditd server.
It can be configured to monitor files and syscalls.

\acmlisting[label=auditd install,caption=auditd install]{./logging/auditd/scripts/audit_install.sh}

The user space auditing commands, can be used to configure logs.
Audit can stores its rules in /etc/audit/audit.rules or in files inside /etc/audit/audit.d/
The syntax for these files is the same as the user space commands

\acmlisting[label=auditd install,caption=auditd install]{./logging/auditd/scripts/audit_rules.sh}

Viewing the auditlog can be done in a few ways:

\begin{enumerate}
	\item aureport --- query logs for a specific event
	\item ausearch --- view a summary of recent events
	\item syslog --- view logs typically stored in /var/log/audit/audit.log
\end{enumerate}
